\documentclass[11pt]{letter}

\usepackage[hmargin={1.0in,1.0in},%
            vmargin={1.0in,1.0in},%
            nohead,%
            nofoot,%
            ]{geometry}                                 % the page layout without fancyhdr
\pagestyle{empty}

\begin{document}
\address{Ross Parker \\
Department of Mathematics \\
Southern Methodist University \\
Dallas, TX 75275 \\
\texttt{rhparker@smu.edu}}%
\signature{Ross Parker}
\begin{letter}{Editor, Physical Review A}

\opening{Dear Editor,}

On behalf of my co-author, Alejandro Aceves, I would like to submit revisions to the article ``Standing Wave Solutions in Twisted Multicore Fibers'' for consideration of publication in Physical Review A. All of the suggestions for improvement from both reviewer have been systematically taken into careful consideration and incorporated into the revision, as noted below. Physical interpretations have been given for all variables and parameters in the model. In addition, two figures have been added (figures 4 and 5) to motivate the discussion of AB suppression for the $N$ even and $N$ odd cases. All figures have been improved for readability. The portions of the manuscript which have been revised are indicated using red text. References to figures and equations are to the revised manuscript; many figure and equation numbers have changed.

Given the improvements made in accordance with the requests of the referees, we hope that you will now find the manuscript to be suitable for publication in Physical Review A. We will be sincerely looking forward to your editorial decision.

First referee
\begin{enumerate}
\item \emph{In Section 2, there is a typo that $0<\phi<2\pi/N$, not $0<\theta<2\pi/N$.} This has been fixed. \emph{And the authors only study the standing wave solutions of $\phi=\pi/N$. How about N is not a integer? e.g., $\phi=1.5*\pi/N$, with $N$ as a integer, can we obtain the similar results like Section 3?} The first example (Figure 3) shows $\phi = 0.25$, which has no relation to $N$. The text now makes this more clear. In addition, we have added a figure showing the amplitude of the node opposite the peak intensity node vs. $\phi$ for $N=6$ (Figure 5, left panel) and $N=7$ (Figure 5, right panel). This suggests that the AB suppression occurs when $\phi = \pi/N$ for $N$ even, independent of the coupling coefficient $k$. This also suggests that AB suppression does not occur for $N$ odd if the peak intensity is concentrated at a single node. This motivates the discussion of AB suppression in the next section, where we show that AB suppression is obtained for $N$ even when $\phi = \pi/N$. This also motivates our consideration of a system with two adjacent excited nodes when we show that AB suppression is obtained for $N$ odd when $\phi = \pi/N$.

\item \emph{In Section 3A, when $N$ is even, the authors say that $k_0$ depends on $N$ and $\omega$, what is the relationship between $k_0$ and $N$, please show it numerically. For different $N$ or $\omega$ (fixed one of the parameters), is the relationship similar? Is this a limitation of the numerical parameter continuation with AUTO method? Or can you give a detailed discussion.} The relationship between $k_0$ and $N$ is now shown numerically in Figure 6.

\item \emph{In Section 5, for asymmetric coupling, for certain different $k_{n}$ and $d=-1$, are there asymmetric unstable standing wave solutions? When $d=1$ in Expression (13), are these asymmetric solutions also neutrally stable? Please give a more detailed explanation.} The parameter $d$ has now been scaled out of the equation, as in Eq (2). We also are only considering the defocusing (minus) nonlinearity, as in the symmetric case. (This corresponds to $d=-1$.) For the asymmetric multi-pulse solution in Figure 13, the spectrum is now plotted in Figure 14, as is results from timestepping of a perturbation of the asymmetric solution. These both suggest that these solutions are also neutrally stable.

\end{enumerate}

Second Referee
\begin{enumerate}
\item \emph{In Eq. (1): a) How is the mode amplitude of each waveguide considered, that is how is its mathematical form?, b) Are $c_n$ complex-valued amplitudes of light electric field of each waveguide? Specify. c) What does $d$ is? Specify. d) Do $k$, $\gamma$ and $d$ depend on $\lambda$?} (a) This is mentioned after Eq (1). (b) The physical interpretation of $c_n$ is given after Eq (1). (c) The physical interpretation of $d$ is given after Eq (1). (d) It is now mentioned after Eq (1) that all parameters depend on the wavelength $\lambda$, and in particular the dependence of $\phi$ on $\lambda$ is explicitly shown.

\item \emph{In first page, second column, first paragraph: It says: ``nearest -neighbor coupling'', do you mean ``nearest-neighbor-waveguide coupling''; It says: ``gain or loss at site $n$'', do you mean ``gain or lost of waveguide $n$''. Correct or clarify.} I did mean ``nearest-neighbor-waveguide coupling'' and ``gain or loss at waveguide $n$''. This has been corrected.

\item \emph{In first page, second column, last paragraph: a) it says: ``Hamiltonian case (2)'' but the Hamiltonian is given by equation (3). Correct. b) It says that $\phi$ is a parameter that represents the twist of the fibers and as shown in Fig. 1. On the other hand in Ref. 2, it is defined as phase Peierls; Say how it is related to the twist of the fiber.} (a) equation (2) is a Hamiltonian system with conserved quantity (3). This is now stated more clearly. (b) The text now states that the quantity $\phi$ is the Peierls phase, and a formula for $\phi$ in terms of physical parameters of the sytstem, including the twist rate $\epsilon$ (which is defined before Eq (1)) and the circle radius $R$ is given.

\item \emph{In page, give a cite for the package AUTO.} This has been added.

\item \emph{Make larger the font size of x- and y-ticks and key labels of all figures} This has been done for all figures.

\item \emph{In Fig. 1: a) ``$N$ twisted fibers arranged in a ring'' might mean different than ``twisted multi-core fiber consisting of $N$ waveguides arranged in a ring''. Check and clarify. b) Do the radius of the fiber play a role on the analysis?} (a) The caption of Figure 1 now clearly states that this is a twisted multi-core fiber consisting of $N$ waveguides arranged in a ring. (b) The radius of the fiber $R$ is now depicted in Figure 1, and the dependence of the Peierls phase on $R$ is given in the formula for $\phi$. The radius of the fiber plays a role in the analysis only via $\phi$.

\item \emph{In equation (4), $\omega$ can not be a parameter of frequency, since $c_n$ is a bound or localized state, the spatial variable in the exponential is $z$ and then $\omega$ must have units of the inverse of length. Check and correct. Since $\omega$ is customary used as a frequency parameter, it might be convenient to use another literal to avoid confusion.} $\omega$ is the propagation constant. We have kept the notation $\omega$ to be consistent with what is used for standing waves in other contexts, e.g. DNLS equation.

\item \emph{In Figs 3, 4 and 6: a) In the text, give a detailed explanation of the plots. Explain the corresponding behavior in terms of the values of the parameters of $k$, $omega$ and $phi$. What are the initial conditions for the amplitudes of each node. What is the input wavelength, power of the electric field? What is the excited mode in the fibers? Why do the states behave in the way shown in the plots.  Make a comparison of the curves shown in panels (a) of figures 3 and 4 and with those of Fig 6 (Explain the reason of the similarity of the curves). Explain. b) In panel (a), it might be convenient to give the horizontal axis scale in multiples of $2\pi$. Give units for $z$. c) In the caption, give units for $\omega$ and $k$. d) Discuss about the results for the phases $\theta$ and $\phi$ shown in corresponding set of equations for both the even and odd cases. It might be convenient to show phase plots for respective examples.} (a) Explanations have been added to all plots. The initial condition is mentioned before Figure 4, and we mention that this is the same initial condition for all evolution plots of standing waves. Similarities and differences between plots are mentioned in the text. (b) Horizontal scales in $z$ are given in increments of $2 \pi$ (for standing waves) and $50 \pi$ (for perturbations). (c) Units have been given in captions for all figures. (d) Phase plots are now given for all examples.

\item \emph{After equation (8) it says: ``the system (8) has a solution for sufficiently small k''. How small should $k$ be? Would it be sufficiently small comparing with what quantity?} The statement has been made more clear to say that (by the implicit functiont theorem) there exists a critical value $k_0$ such that the system has a solution for all $k$ with $|k| < k_0$. In Figure 6, the dependence of $k_0$ on $N$ and $\omega$ is computed numerically.

\item \emph{In last equations in page 4, second column, it might be convenient to use another dummy variable for $k$, to avoid confusion with the nearest-neighbor coupling parameter $k$.} This is a good point and an easy source of confusion. The dummy index variable is now $j$, both on page 4 and the corresponding set of equations for $N$ even on page 3.

\item \emph{In last paragraph of page 3, it says that the observation of a dark node for $N=6$ agrees with what was shown in [2]. Give a comparison of both calculations that of Ref [2] and the present one. Explain the considerations or conditions for both calculations and discuss about it.} There is now a comparison between our results and these prior asymptotic results. In particular, the text explains that our results are much more rigorous and hold for all $N$ even.

\item \emph{In last two lines of page 3, define $k_0$.} $k_0$ is now defined as the critical value of $k$ which is a consequence of the implicit function theorem.

\item \emph{In Fig 5: a) define $l^2$ norm; b) In both the plots and caption, give units for $k$, $k_0$, and $\omega$; c) What does $l^2$ norm equals to one mean?, d) In last line of page 3, it says that $k_0$ depends on N, and in the text it says that Fig. 5 suggests that $k_0$ approaches $\omega$/2 as $N$ gets large; however, there is no respective plot for $k_0$ Vs $N$. Show it.} (a) The $l^2$ norm is now defined in the text. (b) Units are given in the captions of all plots. (c) The $l^2$ norm is $\sqrt{\omega}$ when $k = 0$ (the AC limit) since there is only one excited site. For $\omega = 1$ in the figure, the $l^2$ norm will be 1. (d) a plot of $k_0$ vs $N$ has been added (Figure 6).

\item \emph{Discuss in physical terms, and according to the results, the optical Aharonov-Bohm suppression.} This is now discussed in the introduction.

\item \emph{In Eq. (12): define $q$.} $q$ is the discrete wavenumber. This is now indicated in the text before the dispersion relation equation.

\item \emph{In Fig. 7: a) since the eigenvalues $\lambda$ are purely imaginary, it might be no necessary to multiply Re $\lambda$ by $\times10^{-12}$. This also applies to panel (a) of Fig. 12. b) In caption, ``left panel'' is referred twice but ``right panel'' is not. Check.} (a) we are longer multiplying the real part by $1e-12$ since the eigenvalues are purely imaginary. (b) The panels are now correctly captioned in Figure 7.

\item \emph{In Fig. 8. a) Give units for the $z$ variable, and $z$ b) In caption it says that initial condition is obtained by adding 0.01 to the dark node Do you mean initial condition for the amplitude? Why is such added value for? briefly explain.} (a) Units have been added to captions of all plots. (b) The plots have been redone adding 0.05 to the amplitude of dark node, since that better shows the oscillations. The text now explains that this is added to the amplitude, and that this initial condition was chosen for simplicity. Other initial conditions which are close to the unperturbed solution in amplitude and phase produce results which are qualitatively the same. The motivation behind this choice of perturbation is also briefly explained. Since we expect neutral stability and oscillatory behavior of perturbations, any small perturbation will behave equivalently.

\item \emph{In Figs. 9-12: a) Give units for $z$, $k$ and $k_n$.} (a) Units have been added to captions of all plots.

\item \emph{In Fig 10: a) it might be convenient to show standing wave solutions for the asymmetric coupling case with the same parameter values as in Fig. 3 for comparison. In particular, take $k_n=0.25$, for all $n$, except $k_1=0.4$. b) In panel (a), it might be convenient to give the horizontal axis scale in multiples of $2\pi$. Give units for $z$. c) It might be convenient to show the respective phase plots. d) In the text, give a detailed explanation of the plots and make a comparison of the curves shown with those of Fig. 3. e) How a practical configuration twisted fibers with asymmetric coupling can be build. Explain.} (a) The asymmetric coupling case has now be redone with these parameters ($k_1 = 0.4$, $k_n=0.25$ for all other $n$) for better comparison with the symmetric case. (b) Horizontal scales in $z$ are given in increments of $2 \pi$ (for standing waves) and $50 \pi$ (for perturbations). Units are given in the captions of all figures. (c) Phase plots are now given for all standing wave solutions. (d) A brief comparison is now made with the version for uniform $k$ in Figure 3. (e) This is briefly discussed in Section 5.

\item \emph{In the multi-pulses section, it says that Fig. 11 shows two examples of multi-pulses solutions in which the energy is concentrated at multiple nodes. Panel (a) of the Figure shows 2 maxima at node 1 and 5, 4 minima at nodes 2,4,6 and 8, and 2 dark nodes with zero amplitudes. So, the energy is distributed in all nodes and mainly concentrated in 2 of them. This kind of behavior can also be seen in Fig. 6, where the energy is distributed in all nodes but mainly concentrated in nodes 4 and 5. If so, there is not a clear distinction between cases of one- and multi-pulses. Explain and clarify.} The distinction between multi-pulses and the solutions where the energy is concentrated in two adjacent nodes is mentioned in the text. Briefly, solutions where the energy is concentrated in two adjacent nodes behave like single solitons, whereas multi-pulses behave like a collection of solitons. References have been added to a similar distinction in the discrete NLS equation.

\item \emph{Discuss about the nonlinearity of the fiber. How does it play a role in the presented results.} This is briefly discussed in the introduction and conclusion.

\item \emph{It would be very convenient to show plots of a particular configuration for twisted fibers. That is, show results for a corresponding case given all geometric, linear and nonlinear parameters for the ring configuration formed by twisted fibers. Indeed, this would give a sense of the configuration that should be created experimentally.} We now mention in the conclusion that while the paper emphasizes the theory, suitable parameters from experimental realizations suggested by the references should apply to the results here.

\item \emph{Give citation in order of appearance.} Done using \texttt{apsrev4-2} BibTeX style.

\item \emph{Homogenize the format of references and give them as specified by PRA. Revise all bibliography data fields, some are missing or incorrect. Give them.} Bibliographic references should contain all material. Journals abbreviated per PRA style guide. Formatting done with \texttt{apsrev4-2} BibTeX style with \texttt{aps} and \texttt{pra} parameters.

\end{enumerate}

\closing{Sincerely,}

\end{letter}
\end{document}
